\documentclass[11pt]{scrartcl}

\usepackage{fontspec}
\setmainfont{STIX}
%\setmainfont{EB Garamond}
%\setmainfont{Comic Neue}
%\setsansfont{Comic Sans MS}

\usepackage{polyglossia}
\setdefaultlanguage{english}

\usepackage[margin=2.5cm,bottom=3cm]{geometry}

\setlength\parskip{0.5ex}

\usepackage{enumitem}
\setlist{noitemsep,topsep=1ex,parsep=0.25ex,partopsep=0pt}

\usepackage{tikz}
\usetikzlibrary{positioning}
\usetikzlibrary{calc}

\usepackage{amssymb}

\usepackage[export]{adjustbox}
\usepackage{graphicx}
\usepackage{float}
\usepackage{blindtext}
\usepackage[colorlinks=true]{hyperref}
\newcommand{\cminlinett}[1]{\texttt{#1}}
\usepackage{listings}
\lstset{
	basicstyle=\ttfamily,
	breaklines=true,
	breakatwhitespace=true,
	tabsize=5,
	extendedchars=true,
	inputencoding=utf8,
	showstringspaces=false,
	texcl=false,
	captionpos=b,
	columns=fullflexible
}
\newcommand{\commonmarkimage}[2]{
\begin{figure}[H]
	\centering
	\includegraphics[max width=\linewidth]{#1}
	\caption{#2}
\end{figure}
}

\newcommand{\enq}[1]{«#1»}

\author{Dominik Schmidt \textsf{schmidom@student.ethz.ch}}
\title{Scientific Software Management with Gentoo Linux}
\date{\today}

\newcommand{\dg}[1]{\texttt{#1}}

\begin{document}
	\maketitle
	\begin{abstract}
		In this document key problems and approaches in scientific
		software management with Gentoo Linux are discussed.
		In particular, the \dg{.gentoo} standard, a method to distribute software dependencies with publications and projects, is introduced.
		This method is then applied to multiple use-cases, which includes:
		single-purpose-machines\footnote{Docker, Travis CI},
		scientific computing clusters\footnote{Local compute machines, the EULER cluster}
		and virtual machines\footnote{OpenStack}
	\end{abstract}
	\section{Introduction}
		Empirical research, especially in neurosciences, is often based on a complex pipeline of software doing some sort of statistics, and in the context of neurology often also image processing and transformation.
		This software will depend on more software, constructing a usually large dependency graph\footnote{Trees do not suffice, since the dependencies will most likely contain loops, even for simple programs. A basic example are C compilers, that depend on an already compiled libc, forming the cycle compiler $\leftrightarrows$ libc}.
		
		Hence, package managers are employed to do the dependency resolution.
		One of these managers is Portage, used by Gentoo Linux and derivatives.
		Its key-feature is that the packages are not based on binary tarballs but on recipes (called ebuilds) on how to obtain, build and test the software and what dependencies have to be installed prior.
		This has advantages over other approaches:
		\begin{enumerate}
			\item Ebuilds are easy to write, since they are straight-forward text files written in Bash and nice documentation
			\item Ebuilds are usually easily updated to a new version
			\item The resulting programs are optimized by the compiler to run as efficiently as possible on the machine they are deployed on\footnote{Binary distributions like Debian Linux have binaries compiled to run reasonably fast on \emph{all} machines, which might not always be the fastest way for your specific machine}
		\end{enumerate}
		
		To make this Gentoo Linux approach more suitable for scientific software, we designed a way of bundling an ebuild with the software itself and not inside overlays (the .gentoo standard), and applied this standard to multiple use cases.
	\input{DotGentoo.tex}
	\input{UseCases.tex}
	\input{BuildServer.tex}
	\input{BLAS_Lapack.tex}
	\appendix
	\input{BuildServerExamples.tex}
\end{document}
